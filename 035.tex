\section{Relacyjne bazy danych (6)}

    Bibiloteka ADO.NET udostępnia spójny fundament obsługi różnych rodzajów źródeł danych. W kolejnych latach
	  do platformy .NET nie tylko migrowały uznane technologie mapowania obiektowo-relacyjnego (nHibernate)
	  ale także powstały rozwiązania natywne (Linq2SQL, Entity Framework), które mają duży wpływ na rozwój technologii poza
	  .NET. Równocześnie wraz z udoskonalaniem narzędzi typu ORM obserwuje się odwrót od rozwiązań typu {\tt DataSet}.

\subsection{DataReader (1)}
\label{ereader}

      Przygotować arkusz Excela zawierający dane osobowe (kilka wybranych atrybutów) przykładowej grupy studentów.
\label{ado_excel}	  
      
      Połączyć się do arkusza odpowiednio zainicjowanym połączeniem OleDb ({\tt OleDbConnection}), 
      przeczytać zbiór rekordów za pomocą DataReadera ({\tt OleDbDataReader}) i pokazać je na liście.
            
      [{\bf 1p}]

\subsection{Baza danych (0)}
\label{baza}

\label{ado_dbms}	  

      Przygotować bazę danych Micosoft SQL Server zawierającą dane osobowe i adresy przykładowej grupy studentów.

      Model bazy danych zawiera dwie tabele, tabelę {\bf Student} z polami Imię, Nazwisko, DataUrodzenia oraz
      tabelę {\bf Miejscowosc} z polem Nazwa.
      
      Obie tabele połączone są relacją jeden-do-wielu (jak łączy się tabele relacją jeden-do-wielu?).
	  
	  {\em Uwaga! Do wykonania tego zadania wystarczy darmowy SQL Server Express Edition albo nawet deweloperski
	  SQL Server Local DB}.
      
      [{\bf 0p}] 
      
\subsection{Linq2SQL (1)}

  Zbudować model obiektowy dla bazy danych z zadania \ref{ado_dbms} za pomocą narzędzia {\tt sqlmetal.exe}. 
\label{linq2sql}  
  
  Pokazać w jaki sposób za pomocą Linq2SQL można dodawać, modyfikować i usuwać dane w bazie danych.
  W szczególnosci pokazać jak w jednym bloku kodu dane do {\bf obu} tabel - kod powinien dodać do bazy {\bf nową} miejscowość i {\bf nowego} studenta
  z tej nowej miejscowości. 
    
  [{\bf 1p}]

\subsection{Entity Framework (1)}

  Powtórzyć poprzednie zadanie w technologii Entity Framework.
  
  W szczególności - zbudować model obiektowy dla bazy danych z zadania \ref{ado_dbms} ręcznie lub za pomocą inżynierii odwrotnej (Reverse Engineer Code First).

  Pokazać w jaki sposób za pomocą EF można dodawać, modyfikować i usuwać dane w bazie danych.
  W szczególnosci pokazać jak w jednym bloku kodu dane do {\bf obu} tabel - kod powinien dodać do bazy {\bf nową} miejscowość i {\bf nowego} studenta
  z tej nowej miejscowości. 

  [{\bf 1p}]

\subsection{Interfejs użytkownika dla danych (3)}

      Napisać prostą aplikację okienkową, która udostępnia dane z bazy z poprzedniego zadania.
\label{ado_dal}	  
      Aplikacja powinna pozwalać na przeglądanie listy studentów, dodawanie, modyfikację i usuwanie.

      Do dostępu do danych wybrać Linq2SQL lub Entity Framework.

      [{\bf 3p}]
