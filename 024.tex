\section{Rozszerzenia języka C\# 3.0 (10)}

\subsection{Metoda rozszerzająca klasę {\tt System.String} (1)}

  Zaimplementować metodę {\tt bool IsPalindrome()} rozszerzającą klasę {\tt string}. Implementacja powinna być niewrażliwa na białe znaki i znaki
\label{rozszerzenie_string}  
  przestankowe występujące
  wewnątrz napisu ani na wielkość liter.
  Klient tej metody powinien wywołać ją tak:
  \begin{verbatim}
string s = "Kobyła ma mały bok.";
bool ispalindrome = s.IsPalindrome();
  \end{verbatim}

  [{\bf 1p}]

\subsection{LINQ to Objects, sortowanie, filtrowanie (1)}

  Dany jest plik tekstowy zawierający zbiór liczb naturalnych w kolejnych liniach.
\label{linq_to_objects}  
  Napisać wyrażenie LINQ, które odczyta kolejne liczby z pliku i wypisze tylko
  liczby większe niż 100, posortowane malejąco.
  
\begin{verbatim}  
from liczba in [liczby]
  where ...
    orderby ...
      select ...
\end{verbatim}  
  
  Przeformułować wyrażenie LINQ na ciąg wywołań metod LINQ to Objects:

\begin{verbatim}  
[liczby].Where( ... ).OrderBy( ... )
\end{verbatim}  

  Czym różnią się parametry operatorów {\bf where/orderby} od parametrów
  funkcji {\bf Where, OrderBy}?

  [{\bf 1p}]

\subsection{LINQ to Objects, grupowanie (1)}

  Dany jest plik tekstowy zawierający zbiór nazwisk w kolejnych liniach.

  Napisać wyrażenie LINQ, które zwróci zbiór {\bf pierwszych} liter nazwisk
\label{linq_grupowanie}  
  uporządkowanych w kolejności alfabetycznej. Na przykład dla zbioru
  (Kowalski, Malinowski, Krasicki, Abacki) wynikiem powinien być zbiór
  (A, K, M).
  
  {\em Wskazówka: zgodnie z tytułem zadania użyć operatora group .. by .. into ... }

  [{\bf 1p}]
 
\subsection{LINQ to Objects, agregowanie (1)}

  Napisać wyrażenie LINQ, które dla zadanego foldera wyznaczy sumę
\label{linq_agregowanie}  
  długości plików znajdujących się w tym folderze. 
  
  Do zbudowania sumy długości plików użyć funkcji {\tt Aggregate}. Listę
  plików w zadanym folderze wydobyć za pomocą odpowiednich metod
  z przestrzeni nazw {\tt System.IO}.

  [{\bf 1p}]

\subsection{LINQ to Objects, Join (1)}

  Dane są dwa pliki tekstowe, pierwszy zawierający zbiór danych osobowych postaci
\label{linq_join}  
  (Imię, Nazwisko, PESEL), drugi postaci (PESEL, NumerKonta). Kolejność
  danych w zbiorach jest przypadkowa.
  
  Napisać wyrażenie LINQ, które połączy oba zbiory danych i zbuduje zbiór
  danych zawierający rekordy postaci (Imię, Nazwisko, PESEL, NumerKonta).
  Do połączenia danych należy użyć operatora {\tt join}.
  
  [{\bf 1p}]  
  
\subsection{LINQ to Objects, analiza logów serwera (2)}

  Rejestr zdarzeń serwera IIS 5.5 ma postać pliku tekstowego, w którym każda
\label{linq_logi}  
  linia ma postać:
  
\begin{verbatim}
08:55:36 192.168.0.1 GET /TheApplication/WebResource.axd 200
\end{verbatim}

gdzie poszczególne wartości oznaczają czas, adres klienta, rodzaj żądania HTTP,
nazwę zasobu oraz status odpowiedzi.

  Napisać aplikację która za pomocą jednego (lub wielu) wyrażeń LINQ
  wydobędzie z przykładowego rejestru zdarzeń IIS listę adresów IP trzech
  klientów, którzy skierowali do serwera aplikacji największą liczbę żądań.
  
  Wynikiem działania programu powinien być przykładowy raport postaci:
  
\begin{verbatim}
12.34.56.78 143 
23.45.67.89 113 
123.245.167.289 89
\end{verbatim}
  
  gdzie pierwsza kolumna oznacza adres klienta, a druga liczbę zarejestrowanych żądań.
  
  [{\bf 2p}]

\subsection{Lista obiektów anonimowych (1)}

Listy generyczne ukonkretnieniamy typem elementów:
\label{lista_anonimowych}

\begin{verbatim}
List<int>    listInt;
List<string> listString;...
\end{verbatim}

Z drugiej strony, w C\# 3.0 mamy typy anonimowe, które nie są nigdy jawnie nazwane:

\begin{verbatim}
var item = new { Field1 = "The value", Field2 = 5 };
Console.WriteLine( item.Field1 );
\end{verbatim}

Czy możliwe jest zadeklarowanie i korzystanie z listy generycznej elementów typu anonimowego?

\begin{verbatim}
var item = new { Field1 = "The value", Field2 = 5; }; 
List<?> theList = ?    
\end{verbatim}

W powyższym przykładzie, jak utworzyć listę generyczną, na której znalazłby się element {\bf item}
w taki sposób, by móc następnie do niej dodawać nowe obiekty takiego samego typu?
	 
{\em Obiekty typu anonimowego mają ten sam typ, jesli mają tę samą liczbę składowych tego samego typu
w tej samej kolejnosci.}
	 
  [{\bf 1p}]	 
  
\subsection{Rekursywne anonimowe delegacje (2)}

Cechą charakterystyczną anonimowych delegacji, bez względu na to czy zdefiniowano je przy użyciu słowa
\label{rekursja_anonimowych}
kluczowego {\bf delegate}, czy też raczej jako lambda wyrażenia, jest brak "nazwy", do której można odwołać się
w innym miejscu kodu. 

Zadanie polega na zaproponowaniu takiego tworzenia anonimowych delegacji, żeby w jednym wyrażeniu możliwa
była rekursja. W szczególności, poniższy fragment kodu powinien się kompilować i zwracać wynik zgodny ze specyfikacją.

\begin{verbatim}
List<int> list = new List<int>() { 1,2,3,4,5 };   

foreach ( var item in 
  list.Select( i => [....] ) )   
	
  Console.WriteLine( item );   
}
\end{verbatim}

W powyższym fragmencie kodu, puste miejsce ([....]) należy zastąpić definicją ciała anonimowej delegacji
określonej rekursywnie:

$$
f(i) = \left\{ \begin{array}{ll}
                               1 & i \leq 2 \\
                               f(i-1) + f(i-2) & i > 2
                               \end{array}
                       \right.  
$$

 [{\bf 2p}]	 

{\em Wskazówka} W języku C\# można z powodzeniem zaimplementować operator punktu stałego {\bf Y}, wykorzystywany do definicji funkcji rekurencyjnych. 
Zadanie to można rozwiązać więc definiując taki operator i za jego pomocą implementując funkcję rekurencyjną. Istnieje jednak zaskakujący i o wiele
prostszy sposób rozwiązania wymagający jednak trochę nagięcia specyfikacji. Oba rozwiązania będą przyjmowane.