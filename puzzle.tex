\appendix 

\chapter{Varia}

Niniejszy rozdział zbioru zadań	ma charakter uzupełniający i zawiera zadania dodatkowe, niepunktowane, często o charakterze nieszablonowym, nietypowym, których rozwiązanie
pozwala na pełniejsze zrozumienie wybranych mechanizmów języka i środowiska uruchomieniowego. Zadania z tego rozdziału pochodzą z bloga autora, gdzie były publikowane w latach 2008-2013.

% -----------------------------------------------------------------------------------------------------------------------------------------------------------------
\section{Poziom łatwy}

% ------------------------------------------------------------------------------------
\subsection{Dziwna kolekcja}

Czy to możliwe, że w bibliotece standardowej istnieje kolekcja, która sama tworzy elementy o zadanych kluczach kiedy tylko zostanie o nie poproszona?
Wygląda na to, że tak - poniżej zaprezentowano kod, w którym dopiero co utworzona instancja kolekcji raportuje że zawiera element o losowo wybranym kluczu, 
mimo że taki element nie został tam wczesniej dodany.

\begin{scriptsize}
\begin{verbatim}
namespace ConsoleApplication
{
    class Program
    {
        static void Main( string[] args )
        {
            NameValueCollection Collection = new NameValueCollection();
 
            Console.WriteLine( 
                "Does NameValueCollection magically create items? " +
                Collection["foo"] != null ? " Yes, it does!" : "No, it doesn't." 
                );
        }
    }
}
\end{verbatim}
\end{scriptsize}

Zadaniem Czytelnika jest wskazanie błędu w powyższym kodzie, prowadzącego do takiego nieoczekiwanego zachowania.

% ------------------------------------------------------------------------------------
\subsection{Rekurencyjne zmienne statyczne}

Czy dwie zmienne statyczne, które odwołują się nawzajem do siebie, spowodują powstanie nieskończonej rekursji?

\begin{scriptsize}
\begin{verbatim}
public class A
{
  public static int a = B.b + 1;
}

public class B
{
  public static int b = A.a + 1;
}
 
public class MainClass
{
  public static void Main()
  {
    Console.WriteLine( "A.a={0}, B.b={1}", A.a, B.b );
  }
}
\end{verbatim}
\end{scriptsize}

\subsection{Rozterki kompilatora}

Reguły semantyczne języka muszą precyzyjnie rozstrzygać przypadki "brzegowe". W poniższym przykładzie kompilator ma dwie możliwości - wybrać metodę z klasy bazowej bez konwersji argumentu lub metodę z tej samej klasy
ale z konwersją argumentu. Która reguła obowiązuje w przypadku języka C\#? Czy wybór przeciwnej strategii byłby dopuszczalny?

\begin{scriptsize}
\begin{verbatim}
class A
{
    public void Foo( int n ) 
    {
        Console.WriteLine( "A::Foo" );
    }
}
 
class B : A
{
    /* note that A::Foo and B::Foo are not related at all */
    public void Foo( double n ) 
    {
        Console.WriteLine( "B::Foo" );
    }
}
 
 
static void Main( string[] args )
{
    B b = new B();
    /* which Foo is chosen? */
    b.Foo( 5 );
}
\end{verbatim}
\end{scriptsize}

\subsection{Składowe prywatne}

Czy możliwe jest że klasa {\bf A} ma dostęp do prywatnych składowych klasy {\bf B}? 

"Oczywiście, że nie, to wbrew regule enkapsulacji" - to zwyczajowa odpowiedź. Niemniej, jest co najmniej jeden przypadek, gdy jest to możliwe, co więcej, jest to dość ważna właściwość języka.

Pytanie brzmi więc: w jakich okolicznościach w języku C\# klasa {\bf A} może mieć pełen dostęp do {\bf prywatnych} składowych innej klasy {\bf B}.

\subsection{Nieoczekiwany błąd kompilacji}

Rozważmy poniższy kod

\begin{scriptsize}
\begin{verbatim}
using System;

class Foo
{
   private Foo() { }
}
  
class Program : Foo
{
   static void Main( string[] args )
   {
   }
}
\end{verbatim}
\end{scriptsize}

Próba jego kompilacji kończy się komunikatem

\begin{scriptsize}
\begin{verbatim}
Foo() is inaccessible due to its protection level
\end{verbatim}
\end{scriptsize}

Jest to dość nieoczekiwane, w żadnym miejscu kodu nie ma próby utworzenia nowej instancji typu {\tt Foo}. Ba, w kodzie nie ma w ogóle ani jednego wywołania operatora {\tt new}.
Wydaje się więc, że nie powinno mieć żadnego znaczenia czy konstruktor {\tt Foo} jest dostępny czy nie.

Czytelnik proszony jest o wyjaśnienie powyższego paradoksu.

\subsection{Wywołanie metody na pustej referencji}

Czy możliwe jest wywołanie metody na pustej referencji? Oczywista odpowiedź, to "nie".

Czyżby?

\begin{scriptsize}
\begin{verbatim}
static void Main( string[] args )
{
    Foo _foo = null;
 
    // will throw NullReferenceException
    Console.WriteLine( _foo.Bar() );
 
    Console.ReadLine();
}
\end{verbatim}
\end{scriptsize}

Uzasadnić, że powyższy kod nie musi wcale powodować wyjątku, przeciwnie, może zachować się całkowicie poprawnie i wypisać na konsoli wynik wywołania metody {\tt Bar}.


% ------------------------------------------------------------------------------------------------------------------------------------------------------------
\section{Poziom średniozaawansowany}

% ------------------------------------------------------------------------------------
\subsection{Zamiana wartości dwóch zmiennych}

Następujący kod bywa wykorzystywany w językach C/C++ do zamiany wartości dwóch zmiennych {\bf bez} użycia zmiennej pomocniczej.

\begin{scriptsize}
\begin{verbatim}
int x, y;
 
x ^= y ^= x ^= y;
\end{verbatim}
\end{scriptsize}

Nieoczekiwanie jednak, mimo wspólnych korzeni składni języka, powyższy kod nie działa poprawnie w języku C\#. Zadaniem Czytelnika jest wyjaśnienie dlaczego tak się dzieje.

% ------------------------------------------------------------------------------------
\subsection{Operacje na zbiorach (1)}

Czy Czytelnik potrafi przewidzieć wynik działania poniższego kodu (zawartość która zostanie wypisana na konsoli) {\bf bez} faktycznego uruchomienia?

\begin{scriptsize}
\begin{verbatim}
List<int> list = new List<int>() { 1, 2, 3, 4, 5, 6, 7, 8, 9, 10 };
 
list.FindAll( i => { Console.WriteLine( i ); return i < 5; } );
\end{verbatim}
\end{scriptsize}

\subsection{Operacje na zbiorach (2)}

Po rozwiązaniu poprzedniego zadania Czytelnik z pewnością bez trudu przewidzi również wynik działania poniższego kodu {\bf bez} faktycznego uruchomienia?

\begin{scriptsize}
\begin{verbatim}
List<int> list = new List<int>() { 1, 2, 3, 4, 5, 6, 7, 8, 9, 10 };
 
list.Where( i => { Console.WriteLine( i ); return i < 5; } );
\end{verbatim}
\end{scriptsize}

\subsection{Operacje na zbiorach (3)}

Po rozwiązaniu dwóch poprzednich zadań, przewidzenie wyniku działania poniższego kodu {\bf bez} faktycznego uruchomienia powinno być już łatwe.

\begin{scriptsize}
\begin{verbatim}
List<int> list = new List<int>() { 1, 2, 3 };
 
list.GroupBy ( i => { Console.Write( "X" ); return i; } );
list.ToLookup( i => { Console.Write( "X" ); return i; } );
\end{verbatim}
\end{scriptsize}

% --------------------------------------------------------------------------------------------------------------------------------------------------------------
\section{Poziom trudny}

\subsection{Specyficzne ograniczenie generyczne}

Załóżmy następującą definicję interfejsu generycznego

\begin{scriptsize}
\begin{verbatim}
public interface IGenericInterface<TValue>
{
   ... interface contract
}
\end{verbatim}
\end{scriptsize}

Taki interfejs może być implementowany przez różne klasy z różną wartością argumentu generycznego

\begin{scriptsize}
\begin{verbatim}
class Foo : IGenericInterface<Bar>
{
   ...
}
  
class Bar : IGenericInterface<Baz>
{
   ...
}
\end{verbatim}
\end{scriptsize}

Czy możliwe jest w języku C\# takie ograniczenie generycznego argumentu w definicji interfejsu, żeby jedynym dozwolonym ukonkretnieniem tego argumentu był typ implementujący interfejs?

Mówiąc inaczej, taka i tylko taka definicja typu powinna być dozwolona

\begin{scriptsize}
\begin{verbatim}
class Foo : IGenericInterface<Foo>
{
}
\end{verbatim}
\end{scriptsize}

a taka (i podobne) powinna powodować {\bf błąd kompilacji}

\begin{scriptsize}
\begin{verbatim}
class Foo : IGenericInterface<Bar>
{
}
\end{verbatim}
\end{scriptsize}

\subsection{Zasięg zmiennej w domknięciu}

W poniższym kodzie pętla wewnętrzna tworzy 10 instancji funkcji anonimowych, które "łapią" zmienną lokalną w domknięcie. Wynik działania kodu jest jednak zgoła nieoczekiwany:


\begin{scriptsize}
\begin{verbatim}
// create array of 10 functions
static Func<int>[] constfuncs()
{
    Func<int>[] funcs = new Func<int>[10];
 
    for ( var i = 0; i < 10; i++ )
    {
        funcs[i] = () => i;
    }
 
    return funcs;
}
 
...
 
var funcs = constfuncs();
for ( int i = 0; i < 10; i++ )
    Console.WriteLine( funcs[i]() );
 
// output:
// 10
// 10
// ...
// 10
\end{verbatim}
\end{scriptsize}

Z konstrukcji kodu można bowiem naiwnie oczekiwać, że skoro {\tt i}-ta funkcja powinna, zgodnie z definicja, zwracać wartość {\tt i}. Tak się jednak nie dzieje.

Zadaniem Czytelnika jest nie tylko wyjaśnić powód takiego zachowania się domknięcia, ale również zaproponowanie eleganckiego rozwiązania, w którym nie naruszając zasady 
"{\tt i}-ta funkcja zwraca wartość {\tt i}", wynikiem działania 

\begin{scriptsize}
\begin{verbatim}
for ( int i = 0; i < 10; i++ )
    Console.WriteLine( funcs[i]() );
\end{verbatim}
\end{scriptsize}

będzie

\begin{scriptsize}
\begin{verbatim}
0
1
2
3
4
5
6
7
8
9
\end{verbatim}
\end{scriptsize}
	