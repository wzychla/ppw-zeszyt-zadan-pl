\section{Inne podsystemy Windows (10)}

\subsection{Plik tekstowy na pulpicie, powłoka (1)}

      Napisać program, który na pulpicie bieżącego zalogowanego użytkownika umieści
\label{powloka}	  
      plik tekstowy z bieżącą datą systemową. Następnie plik ten skieruje do wydruku.

      Do pobrania nazwy foldera użyć funkcji {\bf SHGetFolderPath}. Do skierowania dokumentu do wydruku
	  użyć funkcji sterujacej powłoką {\bf ShellExecute}.
      
      [{\bf 1p}]

\subsection{Rozmiar okna w rejestrze (2)}

      Napisać okienkowy program, który zapamięta w rejestrze systemu rozmiary swojego okna.
\label{okno_w_rejestrze}	  
      Rozmiary te powinny być odtwarzane przy każdym uruchomieniu i zapamiętywane przy
      zamykaniu okna programu.
      
      Zaprojektować format zapisu do rejestru. Zapisywać pod kluczem:
      
      HKEY\_CURRENT\_USER$\backslash$Software$\backslash$Programowanie pod Windows$\backslash$...	  
	  
      [{\bf 2p}]

\subsection{Problem golibrody (2)}

      Napisać konsolowy program, który rozwiązuje klasyczny problem golibrody lub problem "palaczy tytoniu" 
\label{golibroda}	  
       za pomocą jeden z metod synchronizacji wątków udostępnianej przez Win32:
	   \begin{itemize}
	   \item muteksy
	   \item semafory
	   \item zdarzenia
	   \end{itemize}

      [{\bf 2p}]
      
\subsection{Internet Explorer jako host dla aplikacji okienkowych (2)}

      Napisać aplikację HTA (HTML Application), która w głównym oknie programu pozwoli wpisać
\label{aplikacja_hta}	  
      imię, nazwisko i datę urodzenia, a po naciśnięciu przycisku "OK" zapisze
      dane do wybranego przez użytkownika pliku tekstowego.
      
      Dlaczego, mimo budowania interfejsu w HTML ta technologia nie
      może być użyta do budowy aplikacji internetowych?
      
      [{\bf 2p}] 
      
\subsection{Informacje o systemie (3)}

      Napisać program do diagnozowania komponentów komputera i systemu operacyjnego. 
\label{informacje_o_systemie}	  
      Raport powinien obejmować m.in.
      \begin{itemize}
      \item Model procesora oraz częstotliwość taktowania 
      \item Ilość pamięci operacyjnej (wolnej, całej)
      \item Wersję systemu operacyjnego wraz z wersją uaktualnienia 
      \item Nazwę sieciową komputera i nazwę aktualnie zalogowanego użytkownika
      \item Ustawienia rozdzielczości i głębi kolorów pulpitu
      \item Listę drukarek podłączonych do systemu
      \item Obecność i numery wersji
            \begin{itemize}
            \item platformy .NET
            \item Internet Explorera
            \item Microsoft Worda
            \end{itemize}
      \end{itemize} 

      [{\bf 3p}] 
      
