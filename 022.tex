\section{.NET $\Leftrightarrow$ Win32, Platform Invoke, COM Interoperability (12)}

    Możliwości platformy .NET byłyby mocno ograniczone, gdyby niemożliwa była współpraca z kodem
	  niezarządzanym. Podobnie jednak jak istnieją dwa różne typy niezarządzanych bibliotek, 
	  bibilioteki natywne i biblioteki COM, tak istnieją dwa różne mechanizmy do współpracy z nimi,
	  {\bf Platform Invoke} do konsumpcji bibliotek natywnych oraz {\bf COM Interoperability} do
	  konsumpcji i produkcji usług COM.  
	  
	  Współpraca z już istniejącym kodem niezarządzanym oznacza tak naprawdę możliwość stopniowego
	  wprowadzania platformy .NET do już istniejących projektów, bez konieczności kosztownego jednorazowego 
	  przenoszenia ich do .NET w całości. To również szansa na wspołpracę .NET zarówno z technologiami, 
	  które z jakichś powodów nigdy nie zostaną przeniesione do kodu zarządza\-ne\-go jak i z innymi 
    technologiami przemysłowymi.
	  
\subsection{P/Invoke, Win32 $\Rightarrow$ .NET (1)}

      Napisać w C\# program, w którym zostanie wywołana funkcja Win32 {\tt GetUserName}, a jej wynik zostanie
\label{pinvoke}	  
      wyprowadzony w oknie informacyjnym, wywołanym przez funkcję Win32 {\tt MessageBox}.
      
      {\em Wskazówka: użyć atrybutów {\tt DllImport}, zadeklarować obie funkcje jako {\tt extern}.}

      [{\bf 1p}]

\subsection{P/Invoke + DLL (2)}

      Napisać w języku C bibliotekę natywną, która udostępnia funkcję {\tt int IsPrimeC}, 
\label{pinvoke_dll}
      sprawdzającą czy podana 32-bitowa liczba jest pierwsza.
      
      Napisać program w C\#, który wywoła tę funkcję z parametrem podanym przez użytkownika z konsoli.

      [{\bf 2p}]

\subsection{P/Invoke + DLL + wskaźniki na funkcje/delegacje (2)}

      Napisać w języku C bibliotekę natywną, która udostępnia funkcję {\tt int ExecuteC} przyjmującą
\label{pinvoke_delegate}	  
      dwa parametry: 32-bitową wartość {\tt n} i wskaźnik na funkcję o sygnaturze {\tt int f(int)}.
      Funkcja {\tt Execute} jako wynik powinna zwracać wartość {\tt f(n)}.

      Napisać program w C\#, który oprócz funkcji {\tt Main} będzie zawierał funkcję {\tt int IsPrimeCs}      
      i który użyje funkcji {\tt ExecuteC} (zastosowanej do funkcji {\tt IsPrimeCs}) do sprawdzenia
      czy podana przez użytkownika z konsoli liczba jest pierwsza.

      Czy możliwe było przeniesienie kodu funkcji {\tt IsPrimeC} z poprzedniego zadania 
      jako funkcji {\tt IsPrimeCs}?

      [{\bf 2p}]

\subsection{COM Interop, COM $\Rightarrow$ .NET, early/late binding (3)}

      To zadanie składa się z 3 części:
\label{cominterop}	  
      
      \begin{enumerate}

      \item
      Napisać bibilotekę COM, która będzie zawierała klasę {\tt PrimeTester}, a w niej metodę {\tt int IsPrime}.
      Napisać skrypt powłoki, w którym ta metoda zostanie wywołana, a wynik pokazany w oknie
      informacyjnym.
      
      {\em Wskazówka: tworzenie bibliotek COM zostało omówione na wykładzie. Zastosować zaproponowaną tam
      metodę: projekt C++ typu {\bf ATL Library}, do niego dodana klasa {\bf ATL COM+ 1.0 Component}.}

      \item
      Napisać program w C\#, w którym zostanie wywołana funkcja {\tt IsPrime} z poprzedniego zadania. 
      Użyć klasy opakowującej (utworzonej automatycznie lub ręcznie).

      \item
      Napisać program w C\#, w którym zostanie wywołana funkcja {\tt IsPrime} z poprzedniego zadania. 
      Zamiast klasy opakowującej użyć refleksji.
      \end{enumerate}
            
      Jakie są wady i zalety wczesnego i późnego wiązania
      (łatwość użycia, bezpieczne typowanie)? Czy użycie wczesnego wiązania jest zawsze możliwe?

      {\em Wskazówka: nauczyć się korzystać z {\tt regsvr32.exe} do rejestrowania i wyrejestrowywania 
      komponentów COM. Nauczyć się korzystać z {\tt tlbimp.exe} do tworzenia klas .NET opakowujących klasy COM.}

      [{\bf 3p}]

\subsection{COM Interop, .NET $\Rightarrow$ COM (4)}

      Napisać w C\# bibliotekę, która będzie zawierała klasę {\tt PrimeTesterCS}, a w niej metodę {\tt int IsPrime}.
\label{cominterop_duzy}	  
      Zarejestrować tę bibliotekę jak bibliotekę COM. Napisać w C++ niezarządzanego klienta COM, 
      zwykłą aplikację konsoli, która skorzysta z tej biblioteki.
      
      Jakie warunki muszą być spełnione, aby klasa .NET mogła być zarejestrowana jako 
      biblioteka COM?

      {\em Wskazówki: 
      
      \begin{enumerate}
      \item 
      Nauczyć się korzystać z atrybutu {\tt GuidAttribute}. 
      Dlaczego warto użyć go do oznaczenia klasy {\tt PrimeTesterCS}? Co stałoby się, gdyby
      nie został on użyty?
      
      \item 
      Nauczyć się korzystać z {\tt sn.exe} do tworzenia plików z sygnaturami cyfrowymi.
      Silnie cyfrowo osygnować bibliotekę, umieszczając odpowiedni atrybut w {\tt AssemblyInfo.cs}.
      Dlaczego trzeba silnie sygnować biblioteki przeznaczone do COM Interop?
            
      \item 
      Nauczyć się korzystać z {\tt gacutil.exe} do zarządzania GAC. Dodać bibliotekę do GAC.
            
      \item 
      Nauczyć się korzystać z {\tt regasm.exe} do rejstrowania bibliotek .NET jako komponentów COM.
      Przy okazji obejrzeć efekt działania {\tt regasm.exe} z parametrem {\tt /regfile}. Zarejestrować bibliotekę
      dla COM Interop.
      
      \item 
      Nauczyć się korzystać z {\tt tlbexp.exe} do eksportowania informacji z bibliotek .NET do
      współpracy z COM. Dlaczego trzeba eksportować informacje o typach do pliku {\tt *.tlb} (TypeLiB)?
      
      \item 
      Nauczyć się korzystać z dyrektywy {\tt \#import} do tworzenia klientów COM w niezarządzanym C++.
      Dlaczego dyrektywy tej należy użyć wskazując jako parametr ścieżkę do pliku {\tt *.tlb}, 
      a nie do biblioteki {\tt *.dll}?
      \end{enumerate}
      }
      
      {\em Uwaga! Ze względu na pewną trudność zadania, za częściowe rozwiązania będą wyjątkowo
      przyznawane punkty pośrednie (między 1 a 4).}
      
      [{\bf 4p}]

