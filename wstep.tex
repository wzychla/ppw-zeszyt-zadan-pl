\chapter*{Wprowadzenie}

Szanowni Państwo!

Niniejszy zbiór zadań przeznaczony jest dla słuchaczy wykładu
{\bf Programowanie pod Windows .NET}, który mam przyjemność prowadzić 
w Instytucie Informatyki Uniwersytetu Wrocławskiego w kolejnych semestrach letnich
od roku akademickiego 2002/2003.
Celem wykładu jest zapoznanie słuchaczy z praktyką programowania systemów operacyjnych rodziny Windows.

Zbiór zadań stanowi uzupełnienie podręcznika,
pozycji {\em Windows oczami programisty} \cite{WZWOP}, 
dostępnej w wersji akademickiej jako skrypt {\em Programowanie pod Windows}.

Zadania zebrano w cztery grupy. Pierwsza część to spojrzenie na fundamenty systemu operacyjnego. 
Druga część to podróż przez historię rozwoju języka C\#, współczesnego referencyjnego języka programowania systemów Windows.
Trzecia część to przegląd biblioteki standardowej platformy .NET. Ostatnia część zawiera zadania dodatkowe, niepunktowane, które pozwalają 
inaczej spojrzeć na wybrane elementy języka i technologii.

Kontynuacją wykładu {\bf Programowanie pod Windows} jest prowdzony w semestrach zimowych
wykład {\bf Projektowanie aplikacji ASP.NET}, który jest 
w całości poświęcony podsystemowi ASP.NET, dedykowanemu rozwijaniu aplikacji internetowych. Z tego też powodu wykład 
{\bf Programowanie pod Windows} świadomie całkowicie pomija ten obszar technologiczny.

Naturalną kontynuacją każdego wykładu technologicznego, w tym tego, jest również wykład dotyczący projektowania obiektowego. Język i technologia są
bowiem tylko sposobem wyrażania aplikacji, a od ich poznania do dobrego projektowania aplikacji jest jeszcze daleka droga. Podobnie -
daleka droga wiedzie od poznania gramatyki i składni języka obcego do pisania w nim książek albo od poznania tego jak mieszać zaprawę murarską do budowania domów.
Dlatego serdecznie zachęcam do wysłuchania wykładów takich jak {\bf Projektowanie obiektowe oprogramowania}, dla których miejsce w którym kończy się {\bf Programowanie pod Windows} jest 
początkiem opowieści o tym jak współcześnie projektuje i wytwarza się oprogramowanie.

\vspace{1cm}
\hfill{\em Wiktor Zychla, luty 2017}\\
\vspace{1cm}
\hfill{\em wzychla@uwr.edu.pl}

