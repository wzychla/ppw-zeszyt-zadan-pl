\section{eXtensible Markup Language (6)}

  Poniższe problemy skomponowano w sposób maksymalnie atomowy, nic nie stoi jednak na przeszkodzie
	aby kilka kolejnych powiązanych zadań połączyć w jednej większej aplikacji.

\subsection{XML (0)}

      Zaprojektować prostą strukturę XML do przechowywania danych o studentach. 
\label{xml1}  
      Każdy student reprezentowany jest {\bf co najmniej} przez podstawowy zbiór atrybutów osobowych,
      ma dwa adresy (stały i tymczasowy) oraz listę zajęć na które uczęszcza wraz z ocenami.
      
      [{\bf 0p}]

\subsection{XSD (1)}

      Schemat struktury z poprzedniego zadania wyrazić w postaci XSD. Zadbać o poprawne opisane reguł walidacji
\label{xsd}	  
      zakresu danych (pewne dane mogą być opcjonalne) i ich zawartości (pewne dane mogą przyjmować 
      wartości o konkretnym formacie).

      [{\bf 1p}]

\subsection{XML + XSD (1)}
\label{XML_XSD}

      Napisać program, który używa zaprojektowanego w poprzednim zadaniu schematu XSD do walidacji wskazanych
\label{xml_xsd}	  
      przez użytkownika plików XML i raportuje ewentualne niezgodności.
      
      [{\bf 1p}]

\subsection{XML - serializacja (1)}

      Napisać prostego klienta struktury XML z zadania~\ref{XML_XSD}, który pliki XML czyta i zapisuje 
\label{xml_serializacja}	  
      mechanizmem serializacji do struktur danych zamodelowanych odpowiednimi atrybutami.
      
      [{\bf 1p}]

\subsection{XML - DOM (1)}

      Napisać prostego klienta struktury XML z zadania~\ref{XML_XSD}, który pliki XML czyta i zapisuje 
\label{xml_dom}	  
      za pomocą modelu DOM ({\tt XmlDocument}).
      
      [{\bf 1p}]

\subsection{XML - strumienie (1)}

      Napisać prostego klienta struktury XML z zadania~\ref{XML_XSD}, który pliki XML czyta i zapisuje 
\label{xml_strumienie}  
      za pomocą mechanizmów strumieniowych ({\tt XmlTextReader, XmlTextWriter}).
      
      [{\bf 1p}]

\subsection{XML - LINQ to XML (1)}

      Napisać wyrażenie LINQ to XML, które z dokumentu XML z poprzednich zadań wybierze dane osobowe studentów o nazwiskach 
\label{xml_linq}	  
      rozpoczynających się na wskazaną literę (wybór litery powinien być możliwy jakkolwiek bez rekompilacji programu). 
      
      [{\bf 1p}]

