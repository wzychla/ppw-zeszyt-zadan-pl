\section{Component Object Model (6)}

Rozwiązanie zadań w tym rozdziale polega na napisaniu programów w
języku C++, korzystając z wbudowanych w Visual Studio szablonów projektów bibliotek COM.

\subsection {Automaryzacja Word (2)}

      Napisać w C/C++ aplikację konsoli, która za pośrednictwem podsystemu COM otworzy nową instancję aplikacji MS Word, a w niej otworzy
\label{com_client}	  
	  nowy dokument,
      do którego wstawi tekst "Programowanie pod Windows". Następnie dokument zostanie zapisany na dysku pod nazwą "ppw.doc".
      
	  Wzorować się na poniższym kodzie:
\begin{scriptsize}	  
\begin{verbatim}
void main()
{
	CoInitialize(NULL);

	CLSID clsid;
	OLECHAR wb[] = L"Word.Application";
	CLSIDFromProgID(wb, &clsid);

	OLECHAR pszCLSID[39];
	StringFromGUID2(clsid, pszCLSID, 39);
	
	char buffer[39];
	wsprintf(buffer, "%S", pszCLSID);
	cout << "CLSID: " << buffer << endl;

	IDispatch* pDispatch;
	CoCreateInstance(clsid, NULL, CLSCTX_SERVER, IID_IDispatch, (void**)&pDispatch);

	DISPID dispid;
	OLECHAR* szMember = L"Visible";

	HRESULT hr = 
	   pDispatch->GetIDsOfNames(IID_NULL, &szMember, 1, LOCALE_SYSTEM_DEFAULT, &dispid);
	if(FAILED(hr))
	    cout << "GetIDsOfNames failed" << endl;
	    cout << "DispID of Visible = " << dispid << endl;

	VARIANTARG test = { VT_BOOL, 0, 0, 0, VARIANT_TRUE };
	DISPID dispidnamed = DISPID_PROPERTYPUT;
	DISPPARAMS param = { &test, &dispidnamed, 1, 1 };

	hr = pDispatch->Invoke(dispid, IID_NULL, LOCALE_SYSTEM_DEFAULT,
		DISPATCH_PROPERTYPUT, &param, NULL, NULL, NULL);
	if(FAILED(hr))
		cout << "Invoke failed" << endl;

	pDispatch->Release();
	CoUninitialize();
}
\end{verbatim}
\end{scriptsize}
	  
      [{\bf 2p}]

\subsection {Serwer COM (3)}

      Przygotować w C++ serwer COM, udostępniający funkcję {\tt int IsPrime( int n )} umożliwiającą sprawdzenie, czy podana liczba jest liczbą 
\label{com_server}	  	  
	  pierwszą. Funkcja powinna zwracać
      zero dla argumentu będącego liczbą złożoną i dowolną niezerową wartość dla argumentu będącego liczbą pierwszą.

      {\em Wskazówka: w Visual Studio należy rozpocząć od projektu C++/ATL Project. 
      Następnie w widoku Class View użyć funkcji Add/ATL COM+ 1.0 Component. 
      Dalsze kroki postępowania zmierzającego do zbudowania serwera COM zostaną
      zaprezentowane na wykładzie.}

      [{\bf 3p}]

\subsection {Klient COM w VBA (1)}

      Napisać w Visual Basic for Applications (język skryptowy Microsoft Office) funkcję wykorzystującą serwer COM z poprzedniego zadania. 
	  Rozwiązanie nie musi posiadać żadnego interfejsu użytkownika do wygodnego wprowadzania argumentu funkcji.

      [{\bf 1p}]

	  